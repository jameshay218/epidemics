\chapter{Synthedemic Modelling}
\label{ch:multi}

Recent work has highlighted the limitations of the single epidemic
based approach in characterising epidemic phenomena, particulary with
regards to viral internet trends.\cite{marily2013, marily2014} A
recent adaptation of the field of \emph{synepidemiology} termed
\emph{synthemics} has been proposed as a potential avenue of further
research. The key challengs of the synthedemic modelling procedure are to
identify the number of underlying epidemics, to identify when these
sub epidemics start, and to identify the type of epidemic model that
best describes each sub epidemic.

\section{Title}
The classic epidemic model brings to mind images of a single curve
with a single peak. When measuring the spread of a single infectious
disease within a closed population, this is often a realistic
characterisation. For example, the number of people in London infected
with a new strain of flu virus might resemble this curve. 
